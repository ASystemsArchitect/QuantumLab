\documentclass[conference]{IEEEtran}
\IEEEoverridecommandlockouts
% The preceding line is only needed to identify funding in the first footnote. If that is unneeded, please comment it out.
\usepackage{cite}
\usepackage{amsmath,amssymb,amsfonts}
\usepackage{algorithmic}
\usepackage{graphicx}
\usepackage{textcomp}
\usepackage{xcolor}
\usepackage{physics}
\def\BibTeX{{\rm B\kern-.05em{\sc i\kern-.025em b}\kern-.08em
    T\kern-.1667em\lower.7ex\hbox{E}\kern-.125emX}}
\begin{document}

\title{Numerical Solution of a Finite Wave Function*}

\author{
	\IEEEauthorblockN{1\textsuperscript{st} Miguel Angel Avila Torres}
	Tunja, Boyacá\\
	miguel.avilat@usantoto.edu.co
}

\maketitle

\begin{abstract}
	This paper describes how to solve the Schrodinger Equation given a valid wave function.
\end{abstract}

\begin{IEEEkeywords}
	Wave function, Schrodinger Equation, Root finding, Integration, Interval, Limits
\end{IEEEkeywords}

\section{Introduction}

\section{Schrodinger Equation}
The time dependent general Schrödinger equation\cite{1} is
\begin{gather}
	i\hbar\frac{d}{dt}\ket{\Psi(t)} = \hat{H}\ket{\Psi(t)}
\end{gather}
Where the Hamiltonian $\hat{H}$ is:
\begin{align}
	\hat{H} &= \hat{T} + \hat{V}\\
	\hat{T} &= \frac{\mathbf{\hat{p}}\cdot\mathbf{\hat{p}}}{2m} = \frac{\hat{p}^2}{2m} = -\frac{\hbar^2}{2m}\nabla^2;~~~~~~\mathbf{\hat{p}}=-i\hbar\nabla\\
	\hat{V} &= V = V(\mathbf{r},t)
\end{align}
being $\hat{V}$ the potential energy, $\hat{H}$ the kinetic energy and $\mathbf{\hat{p}}$ the momentum operator. The equation can be written more completely as
\begin{gather}
	i\hbar\frac{d}{dt}\ket{\Psi(t)} = \left[ -\frac{\hbar}{2m}\nabla^2 + V(\mathbf{r},t) \right] \ket{\Psi(t)}
\end{gather}
Now, if the basis is relative to some point in the space we use\cite{2}
\begin{gather}
	i\hbar\frac{\partial}{\partial t}\Psi(\mathbf{r},t) = \left[\frac{-\hbar^2}{2m}\nabla^2 + V(\mathbf{r},t) \right] \Psi(\mathbf{r},t)
\end{gather}
where instead of a normal derivative we indicate that $\Psi$ depends on $\mathbf{r}$ (a position in the space) so that there is more meaning in the notation.\footnote{note that equation (5) is in terms of kets while (6) is in scalar terms.}

\section{Approaching Numerical Solutions}

\section{Probability in a Closed Interval}

\begin{thebibliography}{00}
	\bibitem{1} R. Shankar, ``Principles of Quantum Mechanics''. Springer, 2012.
	\bibitem{2}	“Schrodinger equation.” http://hyperphysics.phy-astr.gsu.edu/hbase/quantum/Scheq.html (accessed: Sep. 20, 2020).
\end{thebibliography}

\end{document}
